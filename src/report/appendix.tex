\chapter{Equations of Motion}
The equations of motion are of the form
\[
\begin{bmatrix}
{\mathbf I} & {\bm 0} & {\bm 0} \\ {\bm 0} & {\mathbf M} & -{\mathbf G} \\ {\bm 0} & {\bm 0} & {\bm 0}
\end{bmatrix}
\begin{Bmatrix}
\dot{\bm p}\\ \dot{\bm w} \\ \dot{\bm u}
\end{Bmatrix}
+\begin{bmatrix}
{\mathbf V} & -{\mathbf I} & {\bm 0} \\ {\mathbf K} & {\mathbf L} & -{\mathbf F} \\ {\bm 0} & {\bm 0} & {\mathbf I}
\end{bmatrix}
\begin{Bmatrix}
{\bm p}\\ {\bm w}\\ {\bm u}
\end{Bmatrix}
=\begin{bmatrix}
{\bm 0}\\ {\bm 0}\\ {\mathbf I}
\end{bmatrix}
\begin{Bmatrix}
{\bm u}
\end{Bmatrix}
\]
The system is subject to constraints
\[
\begin{bmatrix}
\mathbf{J}_\textrm{h} & {\bm 0} & {\bm 0}\\-{\mathbf J}_\textrm{h}{\mathbf V} & {\mathbf J}_\textrm{h} & {\bm 0}\\ {\bm 0} & {\mathbf J}_\textrm{nh} & {\bm 0}
\end{bmatrix}
\begin{bmatrix}
\dot{\bm p} & {\bm p} \\ \dot{\bm w} & {\bm w} \\ \dot{\bm u} & {\bm u}
\end{bmatrix}
=\begin{bmatrix}
{\bm 0} & {\bm 0}\\{\bm 0} & {\bm 0}\\{\bm 0} & {\bm 0}
\end{bmatrix}
\]

\noindent The full state space equations:
\[
\begin{bmatrix}
{\mathbf E} & {\bm 0} \\ {\bm 0} & {\mathbf I}
\end{bmatrix}
\begin{Bmatrix}
\dot{\bm x}\\ {\bm y}
\end{Bmatrix}
=\begin{bmatrix}
\mathbf{A} & \mathbf{B} \\ \mathbf{C} & \mathbf{D}
\end{bmatrix}
\begin{Bmatrix}
{\bm x}\\ {\bm u}\end{Bmatrix}
\]
\pgfplotstabletypeset[sci,sci zerofill,precision=6,dec sep align,
every head row/.style={output empty row},
every last row/.style={after row={}},
write to macro=\A,
typeset=false]
{unformatted/A.out}
\begin{tiny}
\[
{\mathbf A}=
\left[
\A
\right]
\]
\end{tiny}

\pgfplotstabletypeset[sci,sci zerofill,precision=6,dec sep align,
every head row/.style={output empty row},
every last row/.style={after row={}},
write to macro=\B,
typeset=false]
{unformatted/B.out}
\begin{tiny}
\[
{\mathbf B}=
\left[
\B
\right]
\]
\end{tiny}

\pgfplotstabletypeset[sci,sci zerofill,precision=6,dec sep align,
every head row/.style={output empty row},
every last row/.style={after row={}},
write to macro=\C,
typeset=false]
{unformatted/C.out}
\begin{tiny}
\[
{\mathbf C}=
\left[
\C
\right]
\]
\end{tiny}

\pgfplotstabletypeset[sci,sci zerofill,precision=6,dec sep align,
every head row/.style={output empty row},
every last row/.style={after row={}},
write to macro=\D,
typeset=false]
{unformatted/D.out}
\begin{tiny}
\[
{\mathbf D}=
\left[
\D
\right]
\]
\end{tiny}

\pgfplotstabletypeset[sci,sci zerofill,precision=6,dec sep align,
every head row/.style={output empty row},
every last row/.style={after row={}},
write to macro=\E,
typeset=false]
{unformatted/E.out}
\begin{tiny}
\[
{\mathbf E}=
\left[
\E
\right]
\]
\end{tiny}

\noindent The reduced state space equations:
\[
\begin{Bmatrix}
\dot{\bm x}\\ {\bm y}
\end{Bmatrix}
=\begin{bmatrix}
\mathbf{A} & \mathbf{B} \\ \mathbf{C} & \mathbf{D}
\end{bmatrix}
\begin{Bmatrix}
{\bm x}\\ {\bm u}\end{Bmatrix}
\]
\pgfplotstabletypeset[sci,sci zerofill,precision=6,dec sep align,
every head row/.style={output empty row},
every last row/.style={after row={}},
write to macro=\Am,
typeset=false]
{unformatted/Am.out}
\begin{tiny}
\[
{\mathbf A}=
\left[
\Am
\right]
\]
\end{tiny}

\pgfplotstabletypeset[sci,sci zerofill,precision=6,dec sep align,
every head row/.style={output empty row},
every last row/.style={after row={}},
write to macro=\Bm,
typeset=false]
{unformatted/Bm.out}
\begin{tiny}
\[
{\mathbf B}=
\left[
\Bm
\right]
\]
\end{tiny}

\pgfplotstabletypeset[sci,sci zerofill,precision=6,dec sep align,
every head row/.style={output empty row},
every last row/.style={after row={}},
write to macro=\Cm,
typeset=false]
{unformatted/Cm.out}
\begin{tiny}
\[
{\mathbf C}=
\left[
\Cm
\right]
\]
\end{tiny}

\pgfplotstabletypeset[sci,sci zerofill,precision=4,dec sep align,
every head row/.style={output empty row},
every last row/.style={after row={}},
write to macro=\Dm,
typeset=false]
{unformatted/Dm.out}
\begin{tiny}
\[
{\mathbf D}=
\left[
\Dm
\right]
\]
\end{tiny}

\begin{tiny}
\input{unformatted/Amp.out}
\input{unformatted/Bmp.out}
\input{unformatted/Cmp.out}
\input{unformatted/Dmp.out}
\end{tiny}




% The mass matrix of the system is
% \begin{footnotesize}
% \begin{center}
% \pgfplotstabletypeset[
% begin table=\begin{longtable},
% end table=\end{longtable},
% columns={[index]0,[index]1,[index]2,[index]0,[index]1,[index]2},
% display columns/0/.style={select equal part entry of={0}{2}},
% display columns/1/.style={select equal part entry of={0}{2}},
% display columns/2/.style={select equal part entry of={0}{2}},
% display columns/3/.style={select equal part entry of={1}{2}},
% display columns/4/.style={select equal part entry of={1}{2}},
% display columns/5/.style={select equal part entry of={1}{2}},
% columns/0/.style={int detect, column type=c,column name={Row}},
% columns/1/.style={int detect, column type=c,column name={Column}},
% columns/2/.style={sci,sci zerofill,precision=4,dec sep align,column name={Value}}]
% {unformatted/mass_matrix.out}
% \end{center}
% \end{footnotesize}
% The damping matrix is
% \begin{footnotesize}
% \begin{center}
% \pgfplotstabletypeset[
% begin table=\begin{longtable},
% end table=\end{longtable},
% columns={[index]0,[index]1,[index]2,[index]0,[index]1,[index]2},
% display columns/0/.style={select equal part entry of={0}{2}},
% display columns/1/.style={select equal part entry of={0}{2}},
% display columns/2/.style={select equal part entry of={0}{2}},
% display columns/3/.style={select equal part entry of={1}{2}},
% display columns/4/.style={select equal part entry of={1}{2}},
% display columns/5/.style={select equal part entry of={1}{2}},
% columns/0/.style={int detect, column type=c,column name={Row}},
% columns/1/.style={int detect, column type=c,column name={Column}},
% columns/2/.style={sci,sci zerofill,precision=4,dec sep align,column name={Value}}]
% {unformatted/damping_matrix.out}
% \end{center}
% \end{footnotesize}
% The stiffness matrix is
% \begin{footnotesize}
% \begin{center}
% \pgfplotstabletypeset[
% begin table=\begin{longtable},
% end table=\end{longtable},
% columns={[index]0,[index]1,[index]2,[index]0,[index]1,[index]2},
% display columns/0/.style={select equal part entry of={0}{2}},
% display columns/1/.style={select equal part entry of={0}{2}},
% display columns/2/.style={select equal part entry of={0}{2}},
% display columns/3/.style={select equal part entry of={1}{2}},
% display columns/4/.style={select equal part entry of={1}{2}},
% display columns/5/.style={select equal part entry of={1}{2}},
% columns/0/.style={int detect, column type=c,column name={Row}},
% columns/1/.style={int detect, column type=c,column name={Column}},
% columns/2/.style={sci,sci zerofill,precision=4,dec sep align,column name={Value}}]
% {unformatted/stiffness_matrix.out}
% \end{center}
% \end{footnotesize}
% The velocity matrix is
% \begin{footnotesize}
% \begin{center}
% \pgfplotstabletypeset[
% begin table=\begin{longtable},
% end table=\end{longtable},
% columns={[index]0,[index]1,[index]2,[index]0,[index]1,[index]2},
% display columns/0/.style={select equal part entry of={0}{2}},
% display columns/1/.style={select equal part entry of={0}{2}},
% display columns/2/.style={select equal part entry of={0}{2}},
% display columns/3/.style={select equal part entry of={1}{2}},
% display columns/4/.style={select equal part entry of={1}{2}},
% display columns/5/.style={select equal part entry of={1}{2}},
% columns/0/.style={int detect, column type=c,column name={Row}},
% columns/1/.style={int detect, column type=c,column name={Column}},
% columns/2/.style={sci,sci zerofill,precision=4,dec sep align,column name={Value}}]
% {unformatted/velocity_matrix.out}
% \end{center}
% \end{footnotesize}
% The input force matrix is
% \begin{footnotesize}
% \begin{center}
% \pgfplotstabletypeset[
% begin table=\begin{longtable},
% end table=\end{longtable},
% columns={[index]0,[index]1,[index]2,[index]0,[index]1,[index]2},
% display columns/0/.style={select equal part entry of={0}{2}},
% display columns/1/.style={select equal part entry of={0}{2}},
% display columns/2/.style={select equal part entry of={0}{2}},
% display columns/3/.style={select equal part entry of={1}{2}},
% display columns/4/.style={select equal part entry of={1}{2}},
% display columns/5/.style={select equal part entry of={1}{2}},
% columns/0/.style={int detect, column type=c,column name={Row}},
% columns/1/.style={int detect, column type=c,column name={Column}},
% columns/2/.style={sci,sci zerofill,precision=4,dec sep align,column name={Value}}]
% {unformatted/forcing_matrix.out}
% \end{center}
% \end{footnotesize}
% The input force rate matrix is
% \begin{footnotesize}
% \begin{center}
% \pgfplotstabletypeset[
% begin table=\begin{longtable},
% end table=\end{longtable},
% columns={[index]0,[index]1,[index]2,[index]0,[index]1,[index]2},
% display columns/0/.style={select equal part entry of={0}{2}},
% display columns/1/.style={select equal part entry of={0}{2}},
% display columns/2/.style={select equal part entry of={0}{2}},
% display columns/3/.style={select equal part entry of={1}{2}},
% display columns/4/.style={select equal part entry of={1}{2}},
% display columns/5/.style={select equal part entry of={1}{2}},
% columns/0/.style={int detect, column type=c,column name={Row}},
% columns/1/.style={int detect, column type=c,column name={Column}},
% columns/2/.style={sci,sci zerofill,precision=4,dec sep align,column name={Value}}]
% {unformatted/forcing_rate_matrix.out}
% \end{center}
% \end{footnotesize}
% \begin{footnotesize}
% \begin{center}
% \pgfplotstabletypeset[
% begin table=\begin{longtable},
% end table=\end{longtable},
% columns={[index]0,[index]1,[index]2,[index]0,[index]1,[index]2},
% display columns/0/.style={select equal part entry of={0}{2}},
% display columns/1/.style={select equal part entry of={0}{2}},
% display columns/2/.style={select equal part entry of={0}{2}},
% display columns/3/.style={select equal part entry of={1}{2}},
% display columns/4/.style={select equal part entry of={1}{2}},
% display columns/5/.style={select equal part entry of={1}{2}},
% columns/0/.style={int detect, column type=c,column name={Row}},
% columns/1/.style={int detect, column type=c,column name={Column}},
% columns/2/.style={sci,sci zerofill,precision=4,dec sep align,column name={Value}}]
% {unformatted/jacobian_matrix.out}
% \end{center}
% \end{footnotesize}
